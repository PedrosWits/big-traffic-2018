\section{Introduction}

Automatic number plate recognition (ANPR) cameras are actively employed in urban traffic environments and play an important role in day-to-day intelligent transportation systems. They can be used by Urban Traffic Control and Management (UTMC) entities; by comissioned highway agencies in electronic toll collection; or by law enforcement organisations in the detection of speeding vehicles and validation of number plate registrations. The wide diversity of applications, paired with the large improvement in price-to-performance ratio of ANPR hardware and software systems, has resulted in the increase of installed ANPR cameras across the UK~\cite{SurveyITS2011, EvolutionUTMC2013}.

In the area of Tyne and Wear, United Kingdom, there are over 250 active ANPR cameras. Scans made by these cameras are stored in a central database by the UTMC of Tyne and Wear and then used to compute travel times across particular links of interest in the road network. These are usually major roads that see high volumes of traffic, or road segments more prone to traffic jams. Average journey times can then be conveyed back to the drivers by the way of Variable Message Signs (VMS) or web based applications. ANPR cameras provide additional trip information, besides simple link counts, that has contibuted towards  estimating travel demands and traffic states. extended traditional methods of  travel demands

Even though the cameras only cover a very small subset of the entire road network, the network of cameras provides a valuable time-varying picture of travel demands and traffic state.

In ~\cite{Castillo2008}, the authors used partial trip information from plate number scans to compute travel demands, in the form of a trip matrix, and estimate link flows based on conservatin laws, prior and path flow disaggregation techniques.

% However, Furthermore, ANPR cameras can either be fixed or

Uses of trip data for understandint travel demands / traffic prediction

Studies about best placement of cameras

Number plate data is not as popular, for urban traffic monitoring and prediction , as other sources of data, namely GPS traces or floating vehicle data.

Although a lot of research has gone into 
but, to the best of our knowledge, little research has gone into doing so with number plate data. Therefore, the contributions of this paper are twofold:

\begin{itemize}
  \item
\end{itemize}


% In this paper, we consider the solution of the $N \times
% N$ linear
% system
% \begin{equation} \label{e1.1}
% A x = b
% \end{equation}
% where $A$ is large, sparse, symmetric, and positive definite.  We consider
% the direct solution of (\ref{e1.1}) by means of general sparse Gaussian
% elimination.  In such a procedure, we find a permutation matrix $P$, and
% compute the decomposition
% \[
% P A P^{t} = L D L^{t}
% \]
% where $L$ is unit lower triangular and $D$ is diagonal.
