%%%%%%%%%%%%%%%%%%%%%%%%%%  ltexpprt.tex  %%%%%%%%%%%%%%%%%%%%%%%%%%%%%%%%
%
% This is ltexpprt.tex, an example file for use with the SIAM LaTeX2E
% Preprint Series macros. It is designed to provide double-column output.
% Please take the time to read the following comments, as they document
% how to use these macros. This file can be composed and printed out for
% use as sample output.

% Any comments or questions regarding these macros should be directed to:
%
%                 Donna Witzleben
%                 SIAM
%                 3600 University City Science Center
%                 Philadelphia, PA 19104-2688
%                 USA
%                 Telephone: (215) 382-9800
%                 Fax: (215) 386-7999
%                 e-mail: witzleben@siam.org


% This file is to be used as an example for style only. It should not be read
% for content.

%%%%%%%%%%%%%%% PLEASE NOTE THE FOLLOWING STYLE RESTRICTIONS %%%%%%%%%%%%%%%

%%  1. There are no new tags.  Existing LaTeX tags have been formatted to match
%%     the Preprint series style.
%%
%%  2. You must use \cite in the text to mark your reference citations and
%%     \bibitem in the listing of references at the end of your chapter. See
%%     the examples in the following file. If you are using BibTeX, please
%%     supply the bst file with the manuscript file.
%%
%%  3. This macro is set up for two levels of headings (\section and
%%     \subsection). The macro will automatically number the headings for you.
%%
%%  5. No running heads are to be used for this volume.
%%
%%  6. Theorems, Lemmas, Definitions, etc. are to be double numbered,
%%     indicating the section and the occurence of that element
%%     within that section. (For example, the first theorem in the second
%%     section would be numbered 2.1. The macro will
%%     automatically do the numbering for you.
%%
%%  7. Figures, equations, and tables must be single-numbered.
%%     Use existing LaTeX tags for these elements.
%%     Numbering will be done automatically.
%%
%%  8. Page numbering is no longer included in this macro.
%%     Pagination will be set by the program committee.
%%
%%
%%%%%%%%%%%%%%%%%%%%%%%%%%%%%%%%%%%%%%%%%%%%%%%%%%%%%%%%%%%%%%%%%%%%%%%%%%%%%%%



\documentclass[twoside,leqno,twocolumn]{article}
\usepackage{ltexpprt}
\usepackage{parskip}
\usepackage[inline]{enumitem}
\usepackage{amsmath}
\usepackage[many]{tcolorbox}
\usepackage{graphicx}
\usepackage{subcaption}
\usepackage{makecell}
\usepackage{float}
\usepackage{placeins}
\usepackage{microtype}

\newcommand{\Tau}{\mathrm{T}}

\graphicspath{{./images/}}

\begin{document}


%\setcounter{chapter}{2} % If you are doing your chapter as chapter one,
%\setcounter{section}{3} % comment these two lines out.

\title{\Large Clustering Vehicles based on Trips Identified from Automatic \\ Number Plate Recognition Camera Scans}
\author{Pedro M. Pinto Silva \thanks{School of Computing, Newcastle University, United Kingdom}
\and
Matthew Forshaw\footnotemark[1]
\and
A. Stephen McGough\footnotemark[1]}
\date{}

\maketitle

% Copyright Statement
% When submitting your final paper to a SIAM proceedings, it is requested that you include
% the appropriate copyright in the footer of the paper.  The copyright added should be
% consistent with the copyright selected on the copyright form submitted with the paper.
% Please note that "20XX" should be changed to the year of the meeting.

% Default Copyright Statement
\fancyfoot[R]{\footnotesize{\textbf{Copyright \textcopyright\ 2018 by BigTraffic\\
Unauthorized reproduction of this article is prohibited}}}

% Depending on which copyright you agree to when you sign the copyright form, the copyright
% can be changed to one of the following after commenting out the default copyright statement
% above.

%\fancyfoot[R]{\footnotesize{\textbf{Copyright \textcopyright\ 20XX\\
%Copyright for this paper is retained by authors}}}

%\fancyfoot[R]{\footnotesize{\textbf{Copyright \textcopyright\ 20XX\\
%Copyright retained by principal author's organization}}}


%\pagenumbering{arabic}
%\setcounter{page}{1}%Leave this line commented out.

\begin{abstract} \small\baselineskip=9pt In major cities, government agencies increasingly employ automatic number-plate recognition (ANPR) technology in law enforcement and traffic control. In the Tyne and Wear region (UK) the network of ANPR cameras is used to monitor travel times across sensitive roads. So far, few works have explored the full potential of number-plate scans for analysing individual and collective travel patterns. In this work we present a methodology for deriving trips from vehicle sightings at fixed camera locations. We illustrate the effect of parameters $\tau$ and $\Tau$ on trip discrimination and on the detection of implausible trips. To demonstrate the potential of trip data we apply \emph{k-means} clustering to trips identified from over 40 million plate scans recorded over fifteen weekdays. Results show that whilst private and transit travel modes can begin to be inferred from the resulting clusters, further work needs to be put into developing a more consistent and integrated framework for trip identification in ANPR data.\end{abstract}


% !TEX root = paper.tex
\section{Introduction}

The volume of traffic on our roads has been growing steadily for over 25 years, both in terms of the number of vehicles on the road -- increasing by 40.6\% in the UK~\cite{noVehiocles} -- and the distances covered -- 325.5 billion miles driven in the UK in the year ending September 2017 which is up nearly 30\% in the last 25 years~\cite{distance}. This is placing ever more burden on the road infrastructure along with those who police and manage it. In order to better understand how we can deal with this increase in demand we need to better understand how the road network is being used. By understanding road usage we can better deal with congestion, handle traffic incidents, plan road modifications and deal with illegal acts on the roads.

In a utopian model we would have full disclosure of all journeys made by all vehicles on the road infrastructure. However, this has numerous ethical and technical issues. From an ethical standpoint should we be allowed to know where all vehicles are at any given point in time. From a technical point of view, although every vehicle could be fitted with a GPS tracker -- costly in its own right -- there would still exist the issue of how we would collect and stream all of this data for future processing. Alternatively one can view the problem the other way around and rather than tracking individual vehicles look at collecting information by observing vehicles passing points within the road networks. A prime example of this approach are Automatic Number-plate Recognition (ANPR) cameras. These cameras are a combination of digital camera coupled with Artificial Intelligence to identify number-plates within the image and convert these into strings of characters. ANPR cameras are normally fixed in location\footnote{Although cameras can be in a vehicle and moved from location to location.} able to view all vehicles passing that location.

For ANPR the problem now becomes that of recovering as much information about vehicle's journey as possible from the limited number of observations. ANPR cameras are normally located on major roads and interchanges, however, this only covers a tiny fraction of the road network. We can, though, estimate routes between cameras by understanding the distances between cameras and the most ``sensible'' routes between them. This allows us, given a set of ANPR sightings of the same vehicle, to produce a ``most likely'' route for that journey. It should be noted that we cannot determine the actual start and end of the journey as these will happen in areas not covered by ANPR. It should also be noted that for ethical reasons it is not normal to obtain actual number-plates, but rather the hash of these. Though, for most situations this will suffice.

Once we have a set of sightings of a vehicle using ANPR, we now need to convert these into actual journeys. The first requirement is to identify individual journeys. Although this can't be done with certainty we can apply general rules to distinguish one journey from the next. For example if two sightings are made from ANPR cameras which are connectable by a ``sensible'' route\footnote{Here ``sensible'' implies that a route between cameras A and B would not need to go through a third camera C.} in a time interval which is ``sensible'' then these can be determined to be part of the same journey. However, if the timings between two sightings is significantly longer than what would be expected then this would imply that the vehicle stopped between these two cameras and that the latte sighting is part of a new journey.

The process of journey identification needs to be performed on dirty data which contains numerous impurities which need to be handled. These include:

\begin{itemize}
	\item {\bf Number-plate miss-reads:} Although ANPR cameras have accuracies of around {\color{red}99\%}, miss-reads are possible. This can lead to sightings being missed or vehicles being wrongly sighted in locations.
	\item {\bf Timing errors:} The time-stamps of sightings could be erroneous. The minor side of this is implausible journey times, though, more seriously, this can lead to reordering the set of cameras on a particular journey.
	\item {\bf Clones number-plates:} For various reasons a number-plate may be cloned and used on a different vehicle. This can lead to impossible journeys and journeys that the real vehicle did not make.
\end{itemize}

Once journeys have been identified from the sightings we can then progress by using these journeys to identify higher-order issues within the road network. In this paper we demonstrate how we can use this journey information in order to identify the most likely class each vehicle is a member of. By clustering over such characteristics as how many journeys are made each day, average length of journeys, the number of different ANPR cameras seen in a day and the times when journeys are made we can cluster vehicles into buses, taxis, commuters and delivery vehicles.

The rest of this paper is presented as follows. In Section \ref{s.related} we discuss related work. Section \ref{s.ncl} we presents the ANPR data for the Newcastle area. Our process for identifying individual journeys is presented in Section \ref{s.trips} while Section \ref{s.classification} presents our classification approach. We present results in Section \ref{s.results} before offering conclusions and future directions in Section \ref{s.conclusions}.

\section{Related Work}
\label{s.related}

%Even though the cameras only cover a very small subset of the entire road network, the network of cameras provides a valuable time-varying picture of travel demands and traffic states.

In ~\cite{Castillo2008}, the authors used partial trip information from plate number scans to compute travel demands, in the form of a trip matrix, and estimate link flows from a set of constraints composed of traffic conservation laws, prior knowledge and path flow disaggregation techniques, such as Stochastic User Equilibrium (SUE).

% However, Furthermore, ANPR cameras can either be fixed or

Uses of trip data for understandint travel demands / traffic prediction

Studies about best placement of cameras

Number plate data is not as popular, for urban traffic monitoring and prediction , as other sources of data, namely GPS traces or floating vehicle data.
Previous works trips from GPS traces, .

Although significant research has gone into identifying trips from GPS traces, to the best of our knowledge, little research has gone into doing so with number plate data. Therefore, the contributions of this paper are twofold:

% \begin{enumerate}
%   \item
% \end{itemize}

\section{Tyne and Wear ANPR Data}
\label{s.ncl}
Automatic number plate recognition (ANPR) cameras are actively employed in urban traffic environments and play an important role in day-to-day intelligent transportation systems. They can be used by Urban Traffic Control and Management (UTMC) entities; by comissioned highway agencies in electronic toll collection; or by law enforcement organisations in the detection of speeding vehicles and validation of number plate registrations. The wide diversity of applications, paired with the large improvements in price-to-performance ratios of ANPR hardware and software systems, has resulted in the increase of installed ANPR cameras across the UK~\cite{EvolutionUTMC2013, SurveyITS2011}.

In the area of Tyne and Wear, United Kingdom, there are over 250 active ANPR cameras. Scans made by these cameras are stored in a central database by the UTMC of Tyne and Wear and then used to compute travel times across particular links of interest in the road network. These are usually major roads that see high volumes of traffic, or road segments more prone to traffic jams. Average journey times can then be conveyed back to the drivers by the way of Variable Message Signs (VMS) or web based applications. ANPR cameras provide additional trip information, besides simple link counts, that has

 estimating travel demands and traffic states. extended traditional methods of  travel demands


% !TEX root = paper.tex
\section{Methodology}
\subsection{Tyne and Wear ANPR Data}\label{s.ncl}

\begin{table}[!ht]
\centering
\tabcolsep=0.17cm
\small
\begin{tabular}{c c c c c}
  \hline
  Vehicle & Camera & Timestamp & \thead{Clock \\Error} & Confidence \\
  \hline
  169239 & 1031 & 1454284800.26 &   0 & 100 \\
  12862943 & 18 & 1454284800.97 &   8 &  61 \\
  16243894 & 22 & 1454284801.46 &   6 &  86 \\
  4817789 & 52 & 1454284803.43 &  13 &  94 \\
  5503486 & 110 & 1454284802.19 &  22 &  91 \\
  15244177 & 115 & 1454284802.83 &  18 &  87 \\
   \hline
\end{tabular}
\caption{Sample of number plate data. Clock error is given in milliseconds and confidence as a percentile value.}
\label{t:np_data_example}
%\vspace{-2cm}
\end{table}

Automatic number-plate recognition (ANPR) cameras are actively employed in urban traffic environments and play an important role in day-to-day intelligent transportation systems. They can be used by government subsidised entities in urban traffic management and control; by commissioned highway agencies in electronic toll collection; or by law enforcement organisations in detecting speeding vehicles and validating number plate registrations. The wide diversity of applications, paired with the large improvements in price-to-performance ratios of ANPR hardware and software systems, has resulted in increased investments of ANPR cameras for urban environments~\cite{EvolutionUTMC2013, SurveyITS2011}.

\begin{figure*}[t]
\centering
\begin{subfigure}[t]{.48\textwidth}
  \centering
  \includegraphics[width=.9\linewidth]{observations_per_day.pdf}
  \caption{Total number of scans recorded per day in Tyne and Wear. There is a clear seasonal effect caused by decreasing traffic demands at weekends and increasing traffic volume during weekdays.}
  \label{fig:observations-per-day}
\end{subfigure}\hfill
\begin{subfigure}[t]{.48\textwidth}
  \centering
  \includegraphics[width=.9\linewidth]{observations_per_day_camera.pdf}
  \caption{Number of scans recorded per ANPR camera and day in Tyne and Wear. Inter-camera variability is observed, as some cameras are located in more traffic intensive road sections than others. Decommissioned or temporarily unavailable cameras (due to loss of power, faulty camera, road closed, etc) are depicted at the bottom of the graph.}
  \label{fig:observations-per-camera-day}
\end{subfigure}
\caption{License plate scans recorded by ANPR cameras during February 2017, in the region of Tyne and Wear, United Kingdom.}
\label{fig:time-series}
\end{figure*}

In the region of Tyne and Wear, United Kingdom, there are over 250 active ANPR cameras. Over 1 million license plate detections are recorded by these cameras every day. Figure~\ref{fig:time-series} shows the number of daily scans recorded over a month (February, 2017). Furthermore, every scan is stored in a central database managed by Urban Traffic Management and Control (UTMC) Tyne and Wear, and used to compute travel times across particular links of interest in the road network. These are usually major roads that see high volumes of traffic, or road segments more prone to traffic jams. Average journey times can then be conveyed back to the drivers by the way of Variable Message Signs (VMS) or web based applications. Figure~\ref{fig:anpr-overview} represents this interaction.


Number plate data, in its essence, is a stream of events, each representing a vehicle observed by one camera at a specific point in time. An excerpt of the data can be found in table~\ref{t:np_data_example}. All number-plates were anonymised by the UTMC through a hashing algorithm before the data was shared. Cameras are uniquely identified by an integer and timestamps are relative to each camera's clock. However, the cameras are connected through a private network which provides clock synchronisation using the Network Time Protocol (NTP). Therefore, the timestamps can be used directly if the synchronisation error is negligible. The following additional information is also captured and provided by each camera:
\begin{enumerate*}[label=(\roman*)]
  \item the clock synchronisation error (milliseconds);
  \item the camera's confidence that the identified number plate is the true number plate (percentage);
  \item the direction of travel, away or towards the camera.
\end{enumerate*}
The confidence in the observation is especially useful as it helps diagnosing license plate recognition errors. On the other hand, the direction of travel is dependent upon the orientation of the camera, which is not provided. Hence, we chose to ignore the latter in this work, and aim to explore this information in future works.


% !TEX root = paper.tex
\subsection{Trip Identification}\label{s.trips}

Let the $i_{th}$ sighting of vehicle \emph{k} be defined as the unordered pair:
\begin{equation} \label{e.sighting}
s^{k}_{i} = \{ c, t \},
\end{equation}
where \emph{c} uniquely identifies a camera, and \emph{t} is a scalar representing a point in time (e.g.\ a timestamp).

Let an ordered sequence of sightings of vehicle \emph{k} define the $u_{th}$ trip of vehicle \emph{k}:
\begin{equation} \label{e.trip}
w^{k}_{u} = \left(s^{k}_{(1)}, s^{k}_{(2)}, \dots , s^{k}_{(n)}\right),
\end{equation}
where \( n \) is the degree of the trip, i.e.\ the number of sightings. Moreover, let the corresponding journey time sequence, of degree \(n-1\), be defined as the time difference of consecutive sightings:
\begin{equation} \label{e.journeytime}
jt^{k}_{u} = \left(t^{k}_{(2)} - t^{k}_{(1)}, \ldots, t^{k}_{(n)} - t^{k}_{(n-1)} \right).
\end{equation}
We consider a trip of \emph{$w^k_u$} valid under the following conditions:
\begin{align}
n &\ge 1 , \label{e.trip.constraints.1} \\
\tau_{(i)} &< jt^{k}_{u(i)} < \Tau_{(i)} \ , \ \forall i \in jt^{k}_{u(i)} \label{e.trip.constraints.2},
\end{align}
where $\tau_{(i)}$ and $\Tau_{(i)}$ are the lower and upper bound of the $i_{th}$ element of the journey time sequence.

The first condition~\ref{e.trip.constraints.1} is straightforward and specifies that every identified trip should have at least one sighting. Obviously, vehicles can make trips that do not pass through any ANPR cameras and thus have no associated sightings: $n = 0$. However, this work focus on trips that we can observe and hence we consider that $n > 0$. The second condition~\ref{e.trip.constraints.2} defines a minimum and maximum travel times between consecutive observations. Its purpose is twofold:
\begin{enumerate*}[label=(\roman*)]
  \item first, to allow distinct trips made by the same vehicle to be differentiated. For instance, given two consecutive sightings of \emph{k} three hours apart, we likely want to interpret them as belonging to different trips of \emph{k};
  \item second, it allows implausible trips to be identified. For example, an implausible trip can result from observing \emph{k} at a given camera and then a few seconds later at a second camera, several miles apart. Two explanations are common, either one of the cameras made a detection error, or there is another vehicle with a cloned number-plate travelling on the road network.
\end{enumerate*} Evidently, condition~\ref{e.trip.constraints.2} is only valid for trips of degree two sightings or greater. Nevertheless, trips can easily be differentiated by first sorting sightings by time of occurrence, then calculating the journey time sequence for the entire sequence and finally comparing each element against $\Tau$. An example of a trip identified this way can be seen in Figure~\ref{fig:trip-example}.

The simplest approach to choosing the value of $\Tau$ is to pick a fixed empirical value, such as 5 or 10 minutes. However, if the distance between two cameras is greater than another pair of cameras, then it makes sense that $\Tau$ is relaxed. Similarly, if there is an anomaly in the road network, such as a traffic jam, and the routes connecting the two cameras are affected, then the value of $\Tau$ should also be adapted. Hence, $\Tau$ should be a function of the distance between the two cameras (or, more accurately, of the top n-routes between these) and the distribution of observed journey times. The same rationale can be applied to $\tau$. However, due to time constraints, in this work we simply used several fixed values of $\Tau$ and set $\tau$ to zero. Figure~\ref{fig:length-dist} shows how the number of trips per degree of trip varies by fixing $\Tau$ at different empirical values.

\begin{figure}[t]
  \centering
  \includegraphics[width=1\linewidth]{length-dist.pdf}
  \caption{Distribution of trips per degree of trip.}
  \label{fig:length-dist}
\end{figure}

\begin{figure*}[!ht]%
  \centering
  \begin{subfigure}[c]{.5\textwidth}
    \small
    \tabcolsep=0.09cm
    \begin{tabular}{c c c c c c c}
      \hline
      Vehicle & Camera & Timestamp & Trip & Sighting & \thead{Journey \\Time} & \thead{Trip \\Id}\\
      \hline
      2362920 & 1014 & \makecell{2017-02-01 \\ 00:00:06} &   1 &   1 & NA & 21 \\
      2362920 & 1044 & \makecell{2017-02-01 \\ 00:01:28} &   1 &   2 & 82.38 & 21 \\
      2362920 &  35 & \makecell{2017-02-01 \\ 00:02:32} &   1 &   3 & 63.50 & 21\\
      2362920 &  32 & \makecell{2017-02-01 \\ 00:04:38} &   1 &   4 & 125.95 & 21\\
       \hline
    \end{tabular}
    \label{fig:trip-example-table}
  \end{subfigure}\hfill
  %\qquad
  \begin{subfigure}[c]{.48\textwidth}
    \centering
    \includegraphics[width=1\linewidth]{trip-example.pdf}
    \label{fig:trip-example-map}
  \end{subfigure}\hfill
  \caption{Example of a trip of degree 4. On the left side the corresponding data table is shown. The $u_{th}$ trip of each vehicle is given by the variable \emph{Trip}, whereas the $i_{th}$ sighting is given by variable \emph{Sighting}. The variable \emph{JourneyTime} gives the travel time from the previous to current sighting. Lastly, the variable \emph{TripId} represents the unique sequence of cameras that describes the trip. This allows trips to be grouped and summarised not only in terms of their origins and destination, but also routes. On the right side, the same trip is plotted on a map. The circles represent camera locations whereas the lines represent the fastest driving routes between sightings rather than the true route taken by the vehicle. Even though no routing information is available for each consecutive pair of sightings, the observed journey times can be compared against the distribution of collective journey times to rank the set of most likely routes chosen (which can be determined for instance by Stochastic User Equilibrium~\cite{Castillo2008}).}%
  \label{fig:trip-example}%
  \vspace{-0.40cm}
\end{figure*}
\vspace{-0.25cm}

\subsection{Duplicate scannings}

We need to ensure that every trip of vehicle \emph{k} is unique from all other trips of vehicle \emph{k}. That is, given $W^k$ the set of all valid trips of $k$:

\begin{align}
W^{k} = \left( w^{k}_{(1)}, w^{k}_{(2)}, \ldots, w^{k}_{(N)} \right) \label{e.trip.history},
\end{align}

where \emph{N} is the number of trips of \emph{k}, then there should be no two trips containing the same sighting:

\vspace{-0.5cm}
\begin{align}
s^{k}_{(u,i)} \neq s^{k}_{(v,j)}, \forall u,v &= 1, 2, \ldots, N \ , \ u \neq v,  \label{e.trip.history.constraint} \\
\forall i &= 1, 2, \ldots, n^u \ , \\
\forall j &= 1, 2, \ldots, n^v \  \nonumber
\end{align}
where $s_{(u,i)}^k$ is the $i^{th}$ sighting of the $u^{th}$ trip of vehicle $k$, and $n^u$ is the degree of $u$.

In fact, ANPR cameras can identify the same vehicle multiple times in the same pass, if for instance the vehicle is stopped at a junction or traffic light. Hence, if two sightings occurred at the same location in a very short period of time, then there is a strong possibility that these are duplicate observations. As a simplification, we can assume that a trip should not contain cycles and that no camera should appear twice in the same trip. Yet, this assumption ignores cases where a vehicle is required to correct its route by passing through cameras that have already been registered in that trip. Thus, two sightings of vehicle \emph{k} are different if they were observed:
\begin{enumerate*}[label=(\roman*)]
  \item at two different points in time at different locations;
  \item or at the same location with a time interval greater than $\gamma$.
\end{enumerate*}
Otherwise the two sightings are deemed as duplicates:

\vspace{-0.5cm}
\begin{align} \label{e.sighting.different.2}
 (\ c^{k}_{i} &\ne c^{k}_{j} \, )\ \vee \\
 (\ c^{k}_{i} &= c^{k}_{j} \wedge |t^{k}_{i} - t^{k}_{j}| < \gamma \, )\ \Rightarrow s^{k}_{i} \ne s^{k}_{j}  \ , \ i \ne j \nonumber
\end{align}

where $c_i^k$ and $t_i^k$ are the camera and timestamp of the $i_{th}$ sighting of vehicle $k$.

 Although the estimation of $\gamma$ carries similar considerations and consequences as those of estimating $\Tau$ and $\tau$, most duplicates can be identified in consecutive sightings of the same camera within the same trip. Even though a poor estimation of $\gamma$ also has an impact on error propagation, this decreases substantially after filtering duplicates according to the heuristic above, due to the low occurrence of cycles in trips.
\vspace{-0.25cm}

\subsection{Errors in plate scanning}

ANPR cameras have average accuracy rates of 99.9\% and higher. If we consider that on average 1 to 10 out of 10000 number plate scans are misclassified number-plates then approximately between 200 to 2000 scans everyday are incorrect. These errors propagate and lead to innacuracies in the identification of trips. In fact, misclassifications affect the trip sequences of two vehicles: the true passing vehicle and the vehicle erroneously detected instead. The true vehicle will be missing a sighting in the correspoding trip sequence vector whereas the other vehicle's trip sequence will contain an extra invalid sighting. Moreover, ANPR cameras may exceptionally fail to detect passing vehicles. Even though only the passing vehicle is affected in this case, it's trip sequence vector is nonetheless affected. Therefore, due to their impact in trip identification, it is important to detect and address missing and misclassified scans.

We filer invalid sightings by removing all sightings containing a \emph{confidence} value below $85\%$. However, we do not address missing scans in this work.

\begin{table}[!ht]
\centering
\tabcolsep=0.25cm
\begin{tabular}{c c c c}
  \hline
Tau & Trips & \thead{Average\\Degree} & \thead{Proportion of Trips \\ Degree 1} \\
  \hline
5 min & 13,603,759 & 1.46 & 0.70 \\
7.5 min & 12,394,709 & 1.60 & 0.64 \\
10 min & 11,690,791 & 1.69 & 0.61 \\
15 min & 10,823,333 & 1.83 & 0.57 \\
20 min & 10,305,860 & 1.92 & 0.54 \\
30 min &  9,653,499 & 2.04 & 0.52 \\
   \hline
\end{tabular}
\caption{Overview of trip data for varying values of $\Tau$.}
\label{t:trips-tau}
\vspace{-0.5cm}
\end{table}


% !TEX root = paper.tex
\section{Clustering vehicles}\label{s.classification}

One of the most straightfoward applications of trip data is unsupervised learning. Clustering trips to identify

Usou se os seguintes thresholds ...

Filtrou-se todos os confidence blabla..

Filtrou-se duplicates desta forma

Ignorou-se TimestampError porque..

Calculou-se as seguintes features

TABELA de features exemplo

Correlation between features.. we removed three features because of that

\section{Evaluation}\label{s.results}

TABELA de MEANS para o best k e best (ou baseline) threshold?

TABELA DE DIFERENCAS para o best k entre o baseline threshold e os outros thresholds

TABELA de RESULTADOS POR k (e por threshold ou average dos thresholds?), para determinar o melhor k

INTERPRETACAO DE RESULTADOS


% !TEX root = paper.tex
\section{Conclusion and Future Work}\label{s.conclusions}

Most urban cities in the world employ a network of ANPR cameras that are used for law enforcement as well as traffic monitoring and control. Number plate data collected in the Tyne and Wear area is stored and leveraged by the UTMC for computing average journey times across a selection of sensitive roads. However, number plate data could be used more extensively to identify and study individual and collective travel patterns. In this paper, we have presented a set of definitions and constraints that establish a conceptual foundation for identifying vehicle trips from number plate detections.

We have also identified two parameters, $\tau$ and $\Tau$ as critical in the discrimination of plausible and implausible trips. Hence, future work should first and foremost focus on developing formal methods to estimate these parameters from observed distributions of travel times and by applying knowledge about the structure of the road network. Any errors in trip identification that occur due to poor estimation and filtering methods will propagate and be amplified in posterior analysis done using trip data. Moreover, methods for addressing issues concerning camera performance, namely wrong and duplicate scans, should be further developed and researched.

Once trip data has been computed, a range of interesting applications are available. This work tries to identify groups of vehicles by clustering information about frequency and diversity of travel. By associating a vehicle with a cluster that represents taxis, or home-to-work commuters, one can begin estimating trip mode usage across the city. However, the results presented here could benefit from extra work and further validation. Namely, gaining access to a database that maps plate numbers to vehicle types would not only provide means to better validate the proposed methodology but also enable the application of supervised learning instead. Furthermore, this work can be improved upon by using other data sources as a baseline for evaluating the reliability of extracted trips, such as GPS taxi traces. Finally, future work can focus on using trip data to solve interesting research problems such as:
\begin{enumerate*}[label=(\roman*)]
  \item real-time route recommendation using probabilistic graphical models;
  \item detection of abnormal trip patterns for helping law enforcement in the identification of suspect vehicles or behaviour;
  \item modelling how drivers make routing choices in the presence of anomalies in the road network.
\end{enumerate*}

\subsection*{Acknowledgments}

The authors would like to thank Phil Blythe, Professor of Intelligent Transport Systems at Newcastle University, and Ray King from the Tyne and Wear Urban Traffic and Management Centre for providing guidance and access to the Automatic Number Plate Recognition database.


\begin{thebibliography}{99}

\bibitem{ODMobileData}
Alexander, Lauren, et al. "Origin–-destination trips by purpose and time of day inferred from mobile phone data." Transportation research part c: emerging technologies 58 (2015): 240-250.

\bibitem{Castillo2010}
Castillo, Enrique, et al. {\em Optimal use of plate-scanning resources for route flow estimation in traffic networks}. IEEE Transactions on Intelligent Transportation Systems 11.2 (2010): 380-391.

\bibitem{Castillo2008}
Castillo, Enrique, José María Menéndez, and Pilar Jiménez. {\em Trip matrix and path flow reconstruction and estimation based on plate scanning and link observations}. Transportation Research Part B: Methodological 42.5 (2008): 455-481.

\bibitem{Clustering}
Chen, Huiyu, Chao Yang, and Xiangdong Xu. {\em Clustering Vehicle Temporal and Spatial Travel Behavior Using License Plate Recognition Data}. Journal of Advanced Transportation 2017 (2017).

\bibitem{noVehiocles}
Department for Transport {\em Provisional Road Traffic Estimates Great Britain: October 2016-September 2017}. url (visited January 2017): https://www.gov.uk\-/government/statistics/provisional-road-traffic-estima\-tes-great-britain-october-2016-to-september-2017.


\bibitem{distance}
Department for Transport {\em Vehicle Licensing Statistics: Quarter 3 (Jul-Sep) 2017}. url (visited 31 January 2017): https://www.gov.uk/government/statistics\-/vehicle-licensing-statistics-july-to-september-2017.

\bibitem{EvolutionUTMC2013}
Hamilton, Andrew, et al. {\em The evolution of urban traffic control: changing policy and technology}. Transportation planning and technology 36.1 (2013): 24-43.

\bibitem{FourStepModel}
McNally, Michael G. "The four-step model." Handbook of Transport Modelling: 2nd Edition. Emerald Group Publishing Limited, 2007. 35-53.

\bibitem{Hazelton2012}
Parry, Katharina, and Martin L. Hazelton. {\em Estimation of origin-–destination matrices from link counts and sporadic routing data}. Transportation Research Part B: Methodological 46.1 (2012): 175-188.

\bibitem{ClusteringGPS}
Schüssler, Nadine, and Kay W. Axhausen. {\em Identifying trips and activities and their characteristics from GPS raw data without further information}. Arbeitsbericht Verkehrs-und Raumplanung 502 (2008).

\bibitem{Kmeans}
Steinley, Douglas, and Michael J. Brusco. {\em Choosing the number of clusters in Κ-means clustering}. Psychological Methods 16.3 (2011): 285.

\bibitem{SurveyITS2011}
Zhang, Junping, et al. {\em Data-driven intelligent transportation systems: A survey}. IEEE Transactions on Intelligent Transportation Systems 12.4 (2011): 1624-1639.

% \bibitem{ShortestPath?2015}
% Zhu, Shanjiang, and David Levinson. {\em Do people use the shortest path? An empirical test of Wardrop’s first principle}. PloS one 10.8 (2015): e0134322.

% \bibitem{BANKSMITH}
% R.~E. Bank and R.~K. Smith, {\em General sparse elimination requires no
%   permanent integer storage}, SIAM J. Sci. Stat. Comput., 8 (1987),
%   pp.~574--584.
%
% \bibitem{EISENSTAT}
% S.~C. Eisenstat, M.~C. Gursky, M.~Schultz, and A.~Sherman, {\em
%   Algorithms and data structures for sparse symmetric gaussian elimination},
%   SIAM J. Sci. Stat. Comput., 2 (1982), pp.~225--237.

\end{thebibliography}
\end{document}

% End of ltexpprt.tex
