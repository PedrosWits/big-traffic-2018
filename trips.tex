\section{Identifying Trips In Number Plate Data}



\subsection{Definitions}

The \emph{ith} sighting of vehicle \emph{k} can be defined as:

\begin{equation} \label{e2.1.1}
s^{k}_{i} = \left(camera, time\right)
\end{equation}

Trip
\begin{equation} \label{e2.1.2}
trip^{k}_{u} = \left(s^{k}_{(1)}, s^{k}_{(2)}, \dots , s^{k}_{(n)}\right)
\end{equation}

where \( n \) is the number of sightings, i.e.\ length of the trip.

Trips by the same vehicle can be distinguished

From the \( uth \) trip of vehicle \( k \), we can calculate the corresponding journey time sequence, of length \(n-1\):

\begin{equation} \label{e2.1.3}
journey\ time^{k}_{u} = \left(time^{k}_{2} - time^{k}_{1}, \ldots, time^{k}_{n} - time^{k}_{n-1} \right)
\end{equation}

A trip of length \( \ge 2 \) is valid if all partial journey times are lower than a threshold \( \tau \):

\begin{equation} \label{e2.1.4}
journey\ time^{k}_{u(i)} < \tau \ , \ \forall i \in journey\ time^{k}_{u}
\end{equation}

Realistically there should also be a lower bound for each journey time.

Clock Synchronisation

\subsection{Considerations}



% \begin{Definition}{\rm We describe the two methods in \S 1.2. In \S\ 1.3. we
% discuss
% some remaining details.}
% \end{Definition}
%
% Our purpose here is to examine the nonnumerical complexity of the
% sparse elimination algorithm given in  \cite{BANKSMITH}.
% As was shown there, a general sparse elimination scheme based on the
% bordering algorithm requires less storage for pointers and
% row/column indices than more traditional implementations of general
% sparse elimination.  This is accomplished by exploiting the m-tree,
% a particular spanning tree for the graph of the filled-in matrix.
% Several good ordering algorithms (nested dissection and minimum degree)
% are available for computing $P$  \cite{GEORGELIU}, \cite{ROSE72}.
% Since our interest here does not
% focus directly on the ordering, we assume for convenience that $P=I$,
% or that $A$ has been preordered to reflect an appropriate choice of $P$.
%
% \begin{equation} \label{1.2}
%  L_{k-1}v = c
% \end{equation}
% and setting
% \begin{eqnarray}
% \ell &=& D^{-1}_{k-1}v , \\
% \delta &=& \alpha - \ell^{t} v .
% \end{eqnarray}
%
% \begin{figure}
% \vspace{14pc}
% \caption{This is a figure 1.1.}
% \end{figure}
