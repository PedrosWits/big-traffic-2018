\section{Identifying trips in number plate data}
\label{s.trips}
Trip data, identified from number plate data, has the potential to unlock a number of new applications for urban traffic monitoring and control. In this section we are concerned with the methodology for grouping multiple camera observations of the same vehicle into one or several trips of that vehicle.

The \emph{ith} sighting of vehicle \emph{k} can be defined as an unordered pair:

\begin{equation} \label{e.sighting}
s^{k}_{i} = \{ c, t \}
\end{equation}

where \emph{c} is an integer that uniquely identifies a camera, and \emph{t} is a scalar representing a point in time (e.g.\ a timestamp).

An ordered sequence of sightings of vehicle \emph{k} defines the \emph{uth} trip of \emph{k}:

\begin{equation} \label{e.trip}
w^{k}_{u} = \left(s^{k}_{(1)}, s^{k}_{(2)}, \dots , s^{k}_{(n)}\right)
\end{equation}

where \( n \) is the number of sightings, i.e.\ length of the trip. Moreover, for a given trip, we can calculate the corresponding journey time sequence, of length \(n-1\), as the time differences of consecutive sightings:

\begin{equation} \label{e.journeytime}
jt^{k}_{u} = \left(t^{k}_{(2)} - t^{k}_{(1)}, \ldots, t^{k}_{(n)} - t^{k}_{(n-1)} \right)
\end{equation}

We consider a trip of \emph{k} valid if the following conditions are met:

\begin{align}
n &\ge 1 , \label{e.trip.constraints.1} \\
\tau_{(i)} &< jt^{k}_{u(i)} < \Tau_{(i)} \ , \ \forall i \in jt^{k}_{u(i)} \label{e.trip.constraints.2}
\end{align}

The first condition~\ref{e.trip.constraints.1} is straightforward and specifies that every trip should have at least one sighting. The second condition~\ref{e.trip.constraints.2} defines a minimum and maximum travel times between consecutive observations. Its purpose is twofold:
\begin{enumerate*}[label=(\roman*)]
  \item first, to allow trips made by the same vehicle to be differentiated. For instance, given two consecutive sightings of \emph{k} three hours apart, we want to interpret them as belonging to different trips of \emph{k};
  \item second, it allows unplausible trips to be identified. For example, an unplausible trip can result from observing \emph{k} at a given camera and then a few seconds later at a second camera, several miles apart. Two explanations are common, either one of the cameras made a detection error, or there is another vehicle with a cloned plate number travelling in the road network.
\end{enumerate*} Evidently, condition~\ref{e.trip.constraints.2} is only valid for trips of length two or greater.

Furthermore, if we consider all valid trips of vehicle \emph{k}:

\begin{align}
W^{k} = \left( w^{k}_{(1)}, w^{k}_{(2)}, \ldots, w^{k}_{(N)} \right) \label{e.trip.history}
\end{align}

where \emph{N} is the number of trips of \emph{k}, then there should be no two trips containing the same sighting:

\begin{align}
s^{k}_{(u(i))} \neq s^{k}_{(v(j))}, \forall u,v &= 1, 2, \ldots, N \ , \ u \neq v,  \label{e.trip.history.constraint} \\
\forall i,j &= 1, 2, \ldots, n \ , \ i \neq j \nonumber
\end{align}

where, we consider two sightings of vehicle \emph{k} to be different simply if they were observed at two different points in time:

\begin{equation} \label{e.sighting}
t^{k}_{i} \ne t^{k}_{j} \Rightarrow s^{k}_{i} \ne s^{k}_{j} \ , \ i \ne j
\end{equation}

When identifying trips from GPS traces, . It is easy to see that both $\tau$ and $\Tau$ should be functions of the

but, how to estimate the values of $\Tau$ and $\tau$ remains a. For the rest of this work, we will consider these parameters to be fixed.

\subsection{Errors in plate scanning}


\subsection{Duplicate scannings}

\subsection{Clock synchronisation}
